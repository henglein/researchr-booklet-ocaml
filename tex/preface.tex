\header{Preface}{}{}{Message from the Chairs}
\label{Preface}

\newcommand\person[1]{{#1}}
%\newcommand\person[1]{\emph{#1}}

\noindent

Welcome to POPL 2018, the
\emph{45\textsuperscript{th} ACM SIGPLAN Symposium
on the Principles of Programming Languages}.
%
POPL is a premier forum for the discussion of all aspects
of programming languages and programming systems, welcoming
both theoretical and experimental work on a wide variety of
topics, with an emphasis on work that is principled and enduring.
%
We hope that you will enjoy the proceedings,
including both the talks covering a wide range
of technical topics as well as the ``hallway track''
that presents opportunities for learning from
and interacting with researchers, developers,
and students across the planet.

%%% BEGIN:ANDREW

\paragraph{Technical Program}
%
There were 271 reviewed submissions to POPL 2018.
%
A program committee comprising 52 experts from 15 countries,
aided by 130 external reviewers, did the hard work of
thoughtfully evaluating these submissions.
%
Ultimately the program committee selected 66 papers (24\%)
to be presented at the conference and to appear in the
proceedings.
%
In addition, the technical program includes invited keynotes
by Derek Dreyer, Gordon Plotkin and Sarah Lawsky.

\paragraph{Proceedings of the ACM}
%
All papers accepted for POPL 2018 are also being published as one
of the first issues of a new journal, Proceedings of the ACM on
Programming Languages (PACMPL), a Gold Open Access journal publishing
research on all aspects of programming languages, from design to
implementation and from mathematical formalisms to empirical studies.
The publication of these papers in PACMPL is also responsible for
the change to a larger-font, single-column format and the consequent
increase in the maximum page length of papers.

\paragraph{Reviewing Process}
%
As in previous years, and in keeping with ``Principles of POPL'' (Dreyer et
al.), a light-weight double-blind process was used, in which reviews were
written without exposing authorship to reviewers until the discussion phase.
However, a few changes were made to the reviewing process this year,
some dictated by participation in PACMPL and some more experimental.

The significant change in the former category was that all papers were
accepted on a conditional basis. Each conditionally accepted paper
was assigned a ``shepherd'' from the PC, who worked with the authors
and reviewers to ensure that reviewer feedback was addressed by the
final version of every paper. Every conditionally accepted paper was
eventually revised to the satisfaction of the reviewers.

A second, experimental change was the absence of a physical meeting of
the program committee. Instead, all discussion of papers took place
online. Discussion proceeded primarily on the reviewing system,
the excellent HotCRP system developed and supported by Eddie Kohler.
For a few papers, other media such as phone calls were used to achieve
consensus. The
rationale for this change was that while physical meetings have real
benefits, these benefits have not kept pace with the costs as the PC
has grown in size.  Having the entire PC fly to one spot on the globe
for a one-day meeting requires considerable and energy from program
committee members, and the travel comes with significant monetary and
environmental cost as well. While a face-to-face meeting does promote
engaging discussions and also helps build research community in a way
that is especially useful to junior members of the PC, these positive
effects have diminished as the PC has grown. Having no physical
meeting made it possible to grow the PC to more than 50 members; as
a result, it was also possible to reduce the reviewing load per PC
member slightly from that in recent years.

A poll of the program committee was used to assess this experiment.
Based on the poll, the effect of online discussion was mixed. Most
PC members felt that online discussion made it easier to follow the
arguments raised in discussion but a significant minority felt that
online discussions decreased their level of engagement somewhat. One
helpful addition to the process was the creation of a ``discussion
committee'', one of whose members was assigned to every paper, with
the responsibility of keeping discussion going in a productive
way. Although all discussions were available to all non-conflicted
reviewers, some reviewers missed having the ``global view'' of all
papers to help with their personal calibration. Overall, the committee
divided in half on whether to conduct future POPL PC discussions
online.

A second significant experiment was that the assignment of
reviewers to papers was done in a much more automatic way than
previously. Instead of having all PC members bids on a large number
of papers or specify their topics of expertise according to some
classification scheme, PC members had the option of identifying a
small number ($\sim\!3$) of papers they were genuinely interested in
reviewing. The remainder of their reviewing assignments were done
using the Toronto Paper Matching System (TPMS), an unsupervised
machine learning algorithm. TPMS has reviewers
provide a corpus of their own publications, whose text is compared
against the text of the submissions. A score is generated for each
(reviewer, paper) pair, and reviewing assignments are then constructed
to maximize the total score, subject to constraints such as
avoiding conflicts of interest and equal reviewing load. Based on
the poll of the PC, this approach seems to have been a success.  The
overall response of the PC, was that their assigned papers tended
to be better suited to their expertise than in previous conference
reviewing, and less work was required from PC members. Overall,
reviewers rated themselves as expert on 44\% of the reviews written;
this was exactly the same percentage as for POPL 2017.

\paragraph{Affiliated Events}
%
We are very pleased that POPL 2018 continues the tradition of being enriched
by a host of co-located events.
%
These include three co-located conferences,
%
\begin{itemize}
  \item CPP 2018:   The \emph{7\textsuperscript{th} ACM SIGPLAN Intl. Conference on Certified Programs and Proofs},
  \item PADL 2018:  The \emph{20\textsuperscript{th} Intl. Symposium on Practical Aspects of Declarative Languages}, and
  \item VMCAI 2018: The \emph{19\textsuperscript{th} Intl. Conference on Verification, Model Checking, and Abstract Interpretation}.
\end{itemize}
%
seven co-located workshops focussed on variety of foundational and applied topics,
%
\begin{itemize}
  \item CoqPL 2018: \emph{The 4\textsuperscript{th} Intl. Workshop on Coq for Programming Languages},
  \item NetPL 2018: \emph{The 4\textsuperscript{th} Intl. Workshop on Networking and Programming Languages},
  \item PEPM 2018:  \emph{The ACM SIGPLAN Workshop on Partial Evaluation and Program Manipulation}
  \item OBT 2018:   \emph{Off the Beaten Track}
  \item PriSC 2018: \emph{Principles of Secure Compilation}
  \item PPS 2018:   \emph{Probabilistic Programming Languages, Semantics, and Systems},
\end{itemize}
%
Finally, we are especially happy about two events that
will focus on inspiring senior undergraduate and beginning
graduate students to pursue programming languages and
will help them find ways to become a part of the research
community.
%
\begin{itemize}
\item SRC: \emph{Student Research Competition}, and
\item PLMW: \emph{Programming Languages Mentoring Workshop} at POPL.
\end{itemize}
%
All the above events will have joint lunch and coffee breaks
and the same overall schedule structure so that registrants
at any event can attend talks at the other.

\paragraph{Thanks}
%
POPL has grown and thrived for 45 years primarily thanks to
the vibrant community that supports and sustains it.
%
This community includes the authors and developers who
provide the talks and posters, the attendees -- both local
as well as remote -- who provide a stimulating environment
for discussion and debate.
%
However, we would like to especially recognize the \emph{many}
volunteers who---despite their other onerous commitments---have
taken on new responsibilities to help put together this symposium.
%
We are immeasurably thankful for their dedication, effort and expertise.
%
In particular, we would like to thank
%
the program committee, the external reviewers,
and \person{Jean Yang} and \person{Catalin Hritcu}
and the artifact evaluation committee for their
thorough and thoughtful reviews;
%
\person{Marco Gaboardi} for skilfully organizing
the fantastic TutorialFest and co-ordinating with
all the co-locate events;
%
the members of the POPL steering committee --- especially the current
chair \person{Giuseppe Castagna} and new chair \person{Michael Hicks} ---
for their long term stewardship and dedication to the success
of the symposium;
%
\person{Jan Vitek} for being incredibly generous with his time,
helping us put together the website and for many helpful
suggestions;
%
\person{David Walker}, for liaising with our industrial
partners and arranging the generous support and
sponsorships that make POPL possible;
%
\person{Alexandra Silva}, \person{Steve Zdancewic},
\person{Nadia Polikarpova} and \person{Viktor Vafeiadis}
for organizing PLMW;
%
\person{Benjamin Delaware} for running the Student Research Competition;
%
\person{Jean Yang} and \person{Rohit Singh}, for their many novel innovations
using social media and the web to grow the POPL community and
keep it informed, including the wonderful new ``People of Programming Languages''
interviews;
%
\person{Michael Greenberg} for helping POPL transition into a remote participation
era, that will allow anyone on the planet to engage with the speakers even
if they couldn't be physically present in Los Angeles;
%
% CHECK: José Calderón for managing the process of recording and posting videos
% CHECK: for many of the talks at ICFP and associated events;
%
\person{Dirk Beyer} and Conference Publishing Consulting
for compiling the proceedings;
%
the PACMPL Editorial Board and the staff at ACM
headquarters---especially \person{Philip Wadler}---for guiding
and supporting the transition to PACMPL;
%
\person{Alex Sanchez-Stern} and \person{Jakub Zalewski} for organizing
and leading the team of student volunteers that will be
working to keep everything running like clockwork;
%
\person{John Otero} at ACM for helping us find a sweet
venue in downtown Los Angeles.
%
Finally, we will be forever indebted to
\person{Annabel Satin} and \person{Marta Zampollo},
for being the real force behind the conference
(and other SIGPLAN events!), and for making the
task of organizing POPL 2018 miraculously pleasant.

\paragraph{Supporters}
%
POPL 2018 was made possible by our partners, supporters
and sponsors whose generosity helps the community grow
and thrive by making it possible to keep the cost of
registration affordable, and by providing support for
students who would not have been able to attend without
financial aid.
%
We are also very grateful for the generous support of
ACM and ACM SIGPLAN, including their commitment to
PACMPL's Gold Open Access policy, making high-quality,
peer-reviewed scientific research available without
restrictions on access or (re-)use.

\medskip
Welcome again to POPL 2018. We hope you learn new things,
have fun, meet old friends and colleagues, make new
ones, and go home inspired!

\begin{flushright}
\textit{Ranjit Jhala, University of California, San Diego, USA} \\
POPL 2018 General Chair
\medskip \\
\textit{Andrew Myers, Cornell University, USA} \\
POPL 2018 Program Chair
\medskip \\
\end{flushright}


\newpage
