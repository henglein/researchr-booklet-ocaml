\header{Preface}{}{}{Message from the Chairs}
\label{Preface}


\newcommand\person[1]{{#1}}
%\newcommand\person[1]{\emph{#1}}

\noindent

Welcome to POPL 2019, the
\emph{46\textsuperscript{th} ACM SIGPLAN Symposium
on the Principles of Programming Languages}.
%
POPL is a premier forum for the discussion of all aspects
of programming languages and programming systems, welcoming
both theoretical and experimental work on a wide variety of
topics, with an emphasis on work that is principled and enduring.
%
We hope that you will enjoy the proceedings,
including both the talks covering a wide range
of technical topics as well as the ``hallway track''
that presents opportunities for learning from
and interacting with researchers, developers,
and students across the planet.


\paragraph{Technical Program}
%
There were 267 reviewed submissions to POPL 2019.
%
A program committee comprising 52 experts,
aided by 114 external reviewers, did the hard work of
thoughtfully evaluating these submissions.
%
Ultimately, the program committee selected 77 papers (29\%)
to be presented at the conference and to appear in the
proceedings. 
%
The program is also augmented by three journal papers that were accepted to
the ACM Transactions on Programming Languages and Systems (TOPLAS) during
2018.
%
Finally, the program also includes two invited keynotes, by Mark Harman
and Brigitte Pientka.

All papers accepted for POPL 2019 are being published as an issue of the
Proceedings of the ACM on Programming Languages (PACMPL), a Gold Open Access
journal. PACMPL publishes research on all aspects of programming
languages, from design to implementation and from mathematical formalisms to
empirical studies.

\paragraph{Distinguished Paper Award}

A new addition to POPL 2019 is the introduction of a \emph{Distinguished Paper
Award}. This award highlights papers that the POPL program committee thinks
should be read by a broad audience due to their relevance, originality,
significance and clarity. At most 10\% of the accepted papers of POPL were
eligible for this distinction.

We are happy to announce that the distinguished papers of POPL 2019 are

\begin{itemize}
\item {\em A Domain Theory for Statistical Probabilistic Programming}, by 
Matthijs V\'{a}k\'{a}r, Ohad Kammar, and Sam Staton
\item {\em Gradual Parametricity, Revisited}, by
Mat\'{i}as Toro, Elizabeth Labrada, and \'{E}ric Tanter
\item {\em Structuring the Synthesis of Heap-Manipulating Programs}, by 
Nadia Polikarpova and Ilya Sergey
\item {\em From Fine- to Coarse-Grained Dynamic Information Flow Control and
    Back}, by
Marco Vassena, Alejandro Russo, Deepak Garg, Vineet Rajani, and Deian Stefan
\item {\em $A^2$I: Abstract$^2$ Interpretation}, by 
Patrick Cousot, Roberto Giacobazzi, and Francesco Ranzato
\item {\em Context-, Flow- and Field-Sensitive Data-Flow Analysis using
  Synchronized Pushdown Systems}, by
Johannes Sp\"{a}th, Karim Ali, and Eric Bodden
\end{itemize}

The selection of the distinguished papers was made based on the final versions
of the accepted papers and through a second review process. The Distinguish
Paper Award committee was drawn from the POPL program committee and included
Andreas Abel, Michael Greenberg, Suresh Jagannathan, Peter O'Hearn, Andrew
Tolmach, and Stephanie Weirich (chair).

\paragraph{Reviewing Process}
%
The reviewing process for POPL 2019 followed the guidelines laid out in
``Principles of POPL'' (Dreyer et al), making no major changes to the process
used by POPL 2018. After the reviewing process was complete, PC members
were polled about their experience and all responded, save one member
who left the PC due to personal reasons.

Key features of the POPL 2019 review process include
\begin{itemize}
\item Lightweight double-blind submission. Reviews were written without
  exposing authorship to reviewers until the discussion phase.
\item Automated paper assignment. PC members had the option of identifying a
  \emph{small} number of papers they wanted to review. The
  remainder of their reviewing assignments were done using the Toronto Paper
  Matching System (TPMS), an unsupervised machine learning algorithm supported
  by Laurent Charlin. In the poll, PC members found that their assigned
  papers fit at least as well to their expertise as in previous conference
  reviewing and that less work was required from PC members.
\item Two-phase reviewing. Papers were initially assigned two PC reviewers
  and only allocated additional reviews if the first two reviews were
  positive. In the poll, 41 PC members found the multiple rounds
  useful, whereas three reported that it did not work well.
\item Author response. All authors were given the chance to respond to their
  reviews before the PC discussion period. In the poll, 40 PC members
  reported that the author response had a positive effect on decisions and
  reviews and two reported a negative effect.
\item Online PC meeting: Paper discussions were conducted electronically,
  primarily using the excellent HotCRP system developed and supported by Eddie
  Kohler. (More details below.)
\item Conflicts. PC members were allowed to submit to POPL and all conflicts
  were managed by the HotCRP system. PC papers were marked in the system and
  held to a higher standard than non-PC authored papers. Chair conflicts were
  overseen by Lars Birkedal, who will be the PC chair for POPL 2020.
\item Conditional acceptance. All papers eventually accepted were subjected 
  to a two-round reviewing process.  In the first round they were
  conditionally accepted, and any changes by the authors were reviewed prior to
  final acceptance. Each conditionally accepted paper was assigned a ``shepherd''
  from the PC, who worked with the authors and reviewers to ensure that
  reviewer feedback was addressed by the final version of every paper. Every
  conditionally accepted paper was eventually revised to the satisfaction of
  the reviewers.
\end{itemize}

Following the practice initiated by POPL 2018, all discussion of papers took
place online. The rationale for this change was that while physical meetings
have real benefits, these benefits have not kept pace with the costs as POPL
has grown.  Having the entire PC fly to one spot on the globe for a two-day
meeting requires considerable time and energy from PC members, and the travel
comes with significant monetary and environmental cost.  While a face-to-face
meeting facilitates engaging discussions and strengthens the research
community, especially for junior members of the PC, these positive effects
have diminished as the PC has grown. Having no physical meeting made it
possible to invite a PC of more than 50 members. A larger PC both increases
the coverage of topics among the PC members and ensures that the reviewing
load for individual PC members is manageable.

Overall, the sentiment of the POPL 2019 PC was to continue with this format,
answering the question ``Should we have online PC meetings in the future?''
with 31 answering ``yes'' and seven ``no''.  The PC members also commented in
the poll about the benefits of an online meeting, reiterating the rationale
above and also pointing out a few additional benefits. In particular, some
commented that they found the discussion to be more in depth and that it was
easier to make their opinions heard than in a physical meeting. They also
appreciated the asynchronous nature of the meeting, where decisions were not
biased by a discussion order and commented on the fact that an online meeting
makes it easier for PC members to incorporate discussion into the
feedback for the authors.  The negative comments about the change included the
difficulty of keeping up with several discussions in parallel and the lack of
``global view'' of the program, making it difficult for them to identify
related papers to join in the discussion. In particular, some expressed
concern that the program would become more fragmented and less accessible to
the broader community.

There are concerns that an online discussion decreases the level engagement of
PC members and the results of the poll were mixed on this topic. Equal numbers
of PC members found that that their engagement increased as well as decreased
(12 each way), while over half of the PC members reported no
change. Furthermore, when asked whether the online discussion made it easier
or harder to follow the arguments, 27 members reported a positive effect and
eight members reported a negative effect.  Similarly, PC members thought that
the online meeting helped the review quality: 26 members thought that reviews
were improved by this format and only three thought that that they were
negatively affected.

\paragraph{Affiliated Events}
%% FRITZ: check this I just changed the dates and numbers
%
We are very pleased that POPL 2019 continues the tradition of being enriched
by a host of co-located events.
%
These include three co-located conferences:
%
\begin{itemize}
  \item CPP 2019:   The \emph{8\textsuperscript{th} ACM SIGPLAN Intl. Conference on Certified Programs and Proofs},
  \item PADL 2019:  The \emph{21\textsuperscript{st} Intl. Symposium on Practical Aspects of Declarative Languages}, and
  \item VMCAI 2019: The \emph{20\textsuperscript{th} Intl. Conference on Verification, Model Checking, and Abstract Interpretation}.
\end{itemize}
%
Seven colocated workshops focus on a variety of foundational and applied topics:
%
\begin{itemize}
  \item BEAT 2019: \emph{The 4\textsuperscript{th} Intl. Workshop on Behavioral Types},
  \item CoqPL 2019: \emph{The 5\textsuperscript{th} Intl. Workshop on Coq for Programming Languages},
  \item LAFI 2019: \emph{The 1\textsuperscript{st} Intl. Workshop on Languages for Inference (formerly PPS) },
  \item OPCT 2019:   \emph{The 3\textsuperscript{rd} Seminar on Open Problems in Concurrency Theory},
  \item OBT 2019:   \emph{The 8\textsuperscript{th} Off the Beaten Track Workshop},
  \item PEPM 2019:  \emph{The ACM SIGPLAN Workshop on Partial Evaluation and Program Manipulation}, and
  \item PriSC 2019: \emph{The 3\textsuperscript{rd} Workshop on Principles of Secure Compilation}
\end{itemize}
%
Finally, we are especially happy about two events that
focus on inspiring senior undergraduate and beginning
graduate students to pursue programming languages and
that aim to help them become a part of the research
community.
%
\begin{itemize}
\item SRC: \emph{Student Research Competition}, and
\item PLMW: \emph{Programming Languages Mentoring Workshop} at POPL.
\end{itemize}
%
All the above events have joint lunch and coffee breaks
and the same overall schedule structure so that registrants
at any event can attend talks at the other.

\paragraph{Thanks}
%%%fritz, please edit here
%
POPL has grown and thrived for 46 years primarily thanks to
the vibrant community that supports and sustains it.
%
This community includes the authors and developers who
provide the talks and posters and, most importantly, the attendees---both local
as well as remote---who provide a stimulating environment
for discussion and debate.
%
However, we would like to especially recognize the \emph{many}
volunteers who---despite their other onerous commitments---have
taken on new responsibilities to help put together POPL 2019.
%
We are immeasurably thankful for their dedication, effort and expertise.
%
In particular, we would like to thank
%
the POPL 2019 program committee including the external reviewers and 
the artifact evaluation committee for their
thorough and thoughtful reviews, the efforts and value of which can only be underestimated;
%
\person{Benjamin Delaware} and \person{C\u{a}t\u{a}lin Hri\c{t}cu} for professionally chairing the artifict evaluation committee;
%%
\person{Vasco T.~Vasconcelos} for proposing and vetting venue options, co-organizing and co-ordinating all aspects of POPL 2019 in Cascais, and engaging students, colleagues and supporters;
%
\person{Marco Gaboardi} and \person{Zak Kincaid} for skilfully organizing
the TutorialFest and effectively co-ordinating all the co-located events;
%
\person{David Walker} for liaising with our industrial
partners and securing the generous support and sponsorships that make POPL possible;
%
\person{Josh Ko} for being stunningly responsive 
and conscientious in organizing all aspects of the POPL website;
%
\person{Michael Greenberg} for bringing POPL 2019 to the world by announcing and advertising calls for
submission and participation and bringing the world to POPL 2019 via remote participation; 
%
\person{Niki Vazou} and the selection committee members for running the Student Research Competition on a tight deadline and with great success;
%
\person{Alex Sanchez-Stern} and \person{Carlos M\~{a}o de Ferro} for organizing
and leading the team of student volunteers working to keep everything running like clockwork;
%
%\person{Jakub Zalewski} for organizing video recording for co-located events;
%
\person{Justin Hsu}, \person{Bernardo Toninho},
\person{Nobuko Yoshida} and \person{Steve Zdancewic}
for organizing PLMW, which entices and empowers young researchers to conduct programming language research;
%
the POPL steering committee, especially current
chair \person{Michael Hicks}, for their  
long-term stewardship and dedication to the success
of the symposium;
%
the SIGPLAN executive commmittee, especially current chair \person{Jens Palsberg}, for long-term tutelage and sponsorship of POPL, and the SIGPLAN professional activities committee for supporting affordable participation based on both need and merits;
%
%
%\person{Ranjit Jhala} for passing on valuable POPL 2018 knowledge, not least the source code for this booklet;
%
%
%
% CHECK: José Calderón for managing the process of recording and posting videos
% CHECK: for many of the talks at ICFP and associated events;
%
\person{Dirk Beyer} and Conference Publishing Consulting
for compiling the proceedings;
the staff at ACM Headquarters for providing professional guidance and reviews throughout the process;
% 
the many others who have helped us by giving encouragement, advice and a hand in making POPL 2019 happen. 
%
%the PACMPL Editorial Board and the staff at ACM
%headquarters---especially \person{Philip Wadler}---for guiding
%and supporting the transition to PACMPL;
%
%%
Last but most definitely not least, we will be forever indebted to
\person{Annabel Satin} 
for being the real force and workhorse behind the conference
(and other SIGPLAN events!), and for making the
task of organizing POPL 2019 miraculously doable.

\paragraph{Supporters}
%
POPL 2019 was made possible by our partners, supporters
and sponsors whose generosity helps the community grow
and thrive by making it possible to keep the cost of
registration affordable, and by providing support for
students who would not have been able to attend without
financial aid.
%
We are also very grateful for the generous support of
ACM and ACM SIGPLAN, including their commitment to
PACMPL's Gold Open Access policy, making high-quality,
peer-reviewed scientific research available without
restrictions on access or (re-)use.

\medskip
Welcome again to POPL 2019. We hope you learn new things,
have fun, meet old friends and colleagues, make new
ones, and go home inspired!

\begin{flushright}
\textit{Fritz Henglein, University of Copenhagen} \\
POPL 2019 General Chair
\medskip \\
\textit{Stephanie Weirich, University of Pennsylvania} \\
POPL 2019 Program Chair
\medskip \\
\end{flushright}




\newpage
