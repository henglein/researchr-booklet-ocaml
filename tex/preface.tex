\header{Preface}{}{}{Message from the Chairs}
\label{Preface}


\newcommand\person[1]{{#1}}
%\newcommand\person[1]{\emph{#1}}

\noindent

Welcome to POPL 2019, the
\emph{46\textsuperscript{th} ACM SIGPLAN Symposium
on the Principles of Programming Languages}.
%
POPL is a premier forum for the discussion of all aspects
of programming languages and programming systems, welcoming
both theoretical and experimental work on a wide variety of
topics, with an emphasis on work that is principled and enduring.
%
We hope that you will enjoy the proceedings,
including both the talks covering a wide range
of technical topics as well as the ``hallway track''
that presents opportunities for learning from
and interacting with researchers, developers,
and students across the planet.

%%% BEGIN:ANDREW

\paragraph{Technical Program}
%
There were 267 reviewed submissions to POPL 2019.
%
A program committee comprising 52 experts,
aided by 114 external reviewers, did the hard work of
thoughtfully evaluating these submissions.
%
Ultimately the program committee selected 77 papers (29\%)
to be presented at the conference and to appear in the
proceedings. 
%
The program is also augmented by three journal papers that were accepted to
the ACM Transactions on Programming Languages and Systems (TOPLAS), during
2018.
%
Finally, the technical program also includes two invited keynotes by Mark Harman
and Brigitte Pientka.

All papers accepted for POPL 2019 are being published as an issue of the new
journal, Proceedings of the ACM on Programming Languages (PACMPL), a Gold Open
Access journal publishing research on all aspects of programming languages,
from design to implementation and from mathematical formalisms to empirical
studies.

\paragraph{Distinguished Paper Award}

A new addition to POPL 2019 is the introduction of a ``Distinguished Paper
Award.'' This award highlights papers that the POPL program committee thinks
should be read by a broad audience due to their relevance, originality,
significance and clarity. At most 10\% of the accepted papers of POPL are
eligible for this distinction.

We are happy to announce that the distinguished papers of POPL 2019 are:

\begin{itemize}
\item {\em A Domain Theory for Statistical Probabilistic Programming}, by 
Matthijs Vákár, Ohad Kammar, and Sam Staton
\item {\em Gradual Parametricity, Revisited}, by
Matías Toro, Elizabeth Labrada, and Éric Tanter
\item {\em Structuring the Synthesis of Heap-Manipulating Programs}, by 
Nadia Polikarpova and Ilya Sergey
\item {\em From Fine- to Coarse-Grained Dynamic Information Flow Control and
    Back}, by
Marco Vassena, Alejandro Russo, Deepak Garg, Vineet Rajani, and Deian Stefan
\item {\em $A^2$ I: Abstract$^2$ Interpretation}, by 
Patrick Cousot, Roberto Giacobazzi, and Francesco Ranzato
\item {\em Context-, Flow- and Field-Sensitive Data-Flow Analysis using
  Synchronized Pushdown Systems}, by
Johannes Späth, Karim Ali, and Eric Bodden
\end{itemize}

The selection of the distinguished papers was made based on the final version
of the papers and through a second review process. The Distinguish Paper Award
committee was drawn from the POPL program committee and included Andreas Abel,
Michael Greenberg, Suresh Jagannathan, Peter O'Hearn, Andrew Tolmach, and
Stephanie Weirich (chair).

\paragraph{Reviewing Process}
%
As in previous years, and in keeping with ``Principles of POPL'' (Dreyer et
al.), a lightweight double-blind process was used, in which reviews were
written without exposing authorship to reviewers until the discussion phase.
%
Due to the publication in this journal, all papers were
``conditionally'' accepted, and any changes by the authors were reviewed prior
to publication. Each conditionally accepted paper
was assigned a ``shepherd'' from the PC, who worked with the authors
and reviewers to ensure that reviewer feedback was addressed by the
final version of every paper. Every conditionally accepted paper was
eventually revised to the satisfaction of the reviewers.

%
However, a few changes were made to the reviewing process this year,
some dictated by participation in PACMPL and some more experimental.
%

Following the practice initiated by POPL 2018, all discussion of papers took
place online. Discussion proceeded primarily on the reviewing system, the
excellent HotCRP system developed and supported by Eddie Kohler.  For a few
papers, other media such as phone calls were used to achieve consensus. The
rationale for this change was that while physical meetings have real benefits,
these benefits have not kept pace with the costs as the PC has grown in size.
Having the entire PC fly to one spot on the globe for a one-day meeting
requires considerable energy from program committee members, and the
travel comes with significant monetary and environmental cost as well. While a
face-to-face meeting does promote engaging discussions and also helps build
research community in a way that is especially useful to junior members of the
PC, these positive effects have diminished as the PC has grown. Having no
physical meeting made it possible to grow the PC to more than 50 members; as a
result, it was also possible to reduce the reviewing load per PC member
slightly from that in recent years.

A poll of the program committee was used to assess the online experience and
50 members of the PC participated. 
%
Overall, the sentiment of the PC was to continue with this format, answering
the question ``Should we have online PC meetings in the future?'' with 64\%
yes, 14\% no, and the rest some variant of maybe.  The PC members also
commented in the poll about the benefits of an online meeting, reiterating the
rationale above and also pointing out a few additional benefits. In
particular, they appreciated the asynchronous nature of the meeting, where
decisions were not biased by the order they were discussed. They also commented on
the fact that an online meeting makes it easier for PC members to
incorporate discussion into feedback for the authors.  The negative comments
about the change included a lack of ``global view'' of the program, making it
difficult for them to identify related papers to join in the discussion, and
the difficulty of keeping up with several discussions in parallel. 

There are concerns that an online discussion decreases the level engagement of
PC member, but the results of the poll were mixed. About equal numbers found
that that their engagement increased as well as decreased, and over half of
the PC members reported no change. Furthermore, when asked whether the online
discussion made it easier or harder to follow the arguments, 27 members
reported a positive effect and 8 members reported a negative effect.
Similarly, PC members thought that the online meeting helped the review
quality: 26 members thought that reviews were improved by this format and
only three thought that that they were negatively affected. 

Also following last year, was the use of automation in the the assignment of
reviewers to papers.  Instead of having all PC members bid on a large number
of papers or specify their topics of expertise according to some
classification scheme, PC members had the option of identifying a small number
($\sim\!3$) of papers they were genuinely interested in reviewing. The
remainder of their reviewing assignments were done using the Toronto Paper
Matching System (TPMS), an unsupervised machine learning algorithm supported
by Laurent Charlin. With TPMS, reviewers provide a corpus of their own
publications, whose text is compared against the text of the submissions. A
score is generated for each (reviewer, paper) pair, and reviewing assignments
are then constructed to maximize the total score, subject to constraints such
as avoiding conflicts of interest and equal reviewing load. 

Based on the poll of the PC, this approach seems to have been a success.  The
overall response of the PC was that their assigned papers were at least as
well to their expertise as in previous conference reviewing, and less work was
required from PC members.

\paragraph{Affiliated Events}
%% FRITZ: check this I just changed the dates and numbers
%
We are very pleased that POPL 2019 continues the tradition of being enriched
by a host of co-located events.
%
These include three co-located conferences,
%
\begin{itemize}
  \item CPP 2019:   The \emph{8\textsuperscript{th} ACM SIGPLAN Intl. Conference on Certified Programs and Proofs},
  \item PADL 2019:  The \emph{21\textsuperscript{th} Intl. Symposium on Practical Aspects of Declarative Languages}, and
  \item VMCAI 2019: The \emph{20\textsuperscript{th} Intl. Conference on Verification, Model Checking, and Abstract Interpretation}.
\end{itemize}
%
Seven colocated workshops focus on variety of foundational and applied topics,
%
\begin{itemize}
  \item CoqPL 2019: \emph{The 5\textsuperscript{th} Intl. Workshop on Coq for Programming Languages},
  \item NetPL 2019: \emph{The 5\textsuperscript{th} Intl. Workshop on Networking and Programming Languages},
  \item PEPM 2019:  \emph{The ACM SIGPLAN Workshop on Partial Evaluation and Program Manipulation}
  \item OBT 2019:   \emph{Off the Beaten Track}
  \item PriSC 2019: \emph{The 3rd Workshop on Principles of Secure Compilation}
  \item PPS 2019:   \emph{Probabilistic Programming Languages, Semantics, and Systems},
\end{itemize}
%
Finally, we are especially happy about two events that
focus on inspiring senior undergraduate and beginning
graduate students to pursue programming languages and
that aim to help them become a part of the research
community.
%
\begin{itemize}
\item SRC: \emph{Student Research Competition}, and
\item PLMW: \emph{Programming Languages Mentoring Workshop} at POPL.
\end{itemize}
%
All the above events have joint lunch and coffee breaks
and the same overall schedule structure so that registrants
at any event can attend talks at the other.

\paragraph{Thanks}
%%%fritz, please edit here
%
POPL has grown and thrived for 46 years primarily thanks to
the vibrant community that supports and sustains it.
%
This community includes the authors and developers who
provide the talks and posters, the attendees -- both local
as well as remote -- who provide a stimulating environment
for discussion and debate.
%
However, we would like to especially recognize the \emph{many}
volunteers who---despite their other onerous commitments---have
taken on new responsibilities to help put together this symposium.
%
We are immeasurably thankful for their dedication, effort and expertise.
%
In particular, we would like to thank
%
the program committee, the external reviewers,
and \person{Benjamin Delaware} and \person{C\u{a}t\u{a}lin Hri\c{t}cu}
and the artifact evaluation committee for their
thorough and thoughtful reviews;
%
\person{Marco Gaboardi} for skilfully organizing
the fantastic TutorialFest and co-ordinating with
all the co-located events;
%
the members of the POPL steering committee --- especially the current
chair \person{Giuseppe Castagna} and new chair \person{Michael Hicks} ---
for their long term stewardship and dedication to the success
of the symposium;
%
\person{Jan Vitek} for being incredibly generous with his time,
helping us put together the website and for many helpful
suggestions;
%
\person{David Walker}, for liaising with our industrial
partners and arranging the generous support and
sponsorships that make POPL possible;
%
\person{Alexandra Silva}, \person{Steve Zdancewic},
\person{Nadia Polikarpova} and \person{Viktor Vafeiadis}
for organizing PLMW;
%
\person{Benjamin Delaware} for running the Student Research Competition;
%
\person{Jean Yang} and \person{Rohit Singh}, for their many novel innovations
using social media and the web to grow the POPL community and
keep it informed, including the wonderful new ``People of Programming Languages''
interviews;
%
\person{Michael Greenberg} for helping POPL transition into a remote participation
era, in which anyone on the planet can engage with the speakers even
if they couldn't be physically present in Los Angeles;
%
% CHECK: José Calderón for managing the process of recording and posting videos
% CHECK: for many of the talks at ICFP and associated events;
%
\person{Dirk Beyer} and Conference Publishing Consulting
for compiling the proceedings;
%
the PACMPL Editorial Board and the staff at ACM
headquarters---especially \person{Philip Wadler}---for guiding
and supporting the transition to PACMPL;
%
\person{Alex Sanchez-Stern} and \person{Jakub Zalewski} for organizing
and leading the team of student volunteers
working to keep everything running like clockwork;
%
\person{John Otero} at ACM for helping us find a sweet
venue in downtown Los Angeles.
%
Finally, we will be forever indebted to
\person{Annabel Satin} and \person{Marta Zampollo},
for being the real force behind the conference
(and other SIGPLAN events!), and for making the
task of organizing POPL 2019 miraculously pleasant.

\paragraph{Supporters}
%
POPL 2019 was made possible by our partners, supporters
and sponsors whose generosity helps the community grow
and thrive by making it possible to keep the cost of
registration affordable, and by providing support for
students who would not have been able to attend without
financial aid.
%
We are also very grateful for the generous support of
ACM and ACM SIGPLAN, including their commitment to
PACMPL's Gold Open Access policy, making high-quality,
peer-reviewed scientific research available without
restrictions on access or (re-)use.

\medskip
Welcome again to POPL 2019. We hope you learn new things,
have fun, meet old friends and colleagues, make new
ones, and go home inspired!

\begin{flushright}
\textit{Fritz Henglein, University of Copenhagen} \\
POPL 2019 General Chair
\medskip \\
\textit{Stephanie Weirich, University of Pennsylvania} \\
POPL 2019 Program Chair
\medskip \\
\end{flushright}




\newpage
