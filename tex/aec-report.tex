\header{AEC Chairs' Report}{}{}{Artifact Evaluation Chairs' Report}
\label{aec}

In the programming languages community, artifact evaluation is concerned with
the byproducts of theoretical and applied work. An ``artifact'' is something
intended to support the scientific claims made in a paper. For instance, an
artifact might be a program’s source code, a dataset, a test suite, a mechanized
proof, or a model. ``Evaluation'' is a best-effort attempt to reconcile a paper’s
artifacts with the claims made in the paper. A primary goal of the artifact
evaluation process is to encourage authors to create artifacts that can be
shared and used by others as a basis for new activities. The process has other
benefits as well, such as encouraging authors to be precise in their claims and
providing public recognition for the effort invested in creating artifacts. To
encourage this beneficial behavior, since POPL 2015, authors of accepted papers
have been invited to submit artifacts for evaluation by an Artifact Evaluation
Committee (AEC). In this report we describe the process of artifact evaluation
for POPL 2018, and describe some issues and questions that should be addressed
moving forward.

Members of the POPL Program Committee were invited to nominate senior PhD
students, PostDocs, researchers, or themselves to serve on the AEC.
Participation in the AEC can provide researchers with useful insight into the
value of artifacts and the process of artifact evaluation, and also helps
establish community norms for artifacts. Starting from the nominations and after
some targeted search aimed at balancing the AEC, the chairs formed a committee
of 26 members.

After the POPL 2018 submission decisions were announced, authors of accepted
papers were invited to submit artifacts for evaluation. By design, the artifact
evaluation process had no effect on which papers were chosen to appear at POPL.
Authors had one week to respond to the call for artifacts and submit their
artifact for evaluation. The submission guidelines asked that the authors create
a single web page that contains the artifact and instructions for installing and
using the artifact, while allowing for some exceptional cases. Each artifact was
accompanied by the accepted version of its associated paper so that the AEC
could evaluate each artifact against its paper’s claims. A record number of 40
artifacts were submitted for evaluation out of 66 papers accepted this year
(61\%, also a record). For comparison, POPL 2017 had 31 artifacts submitted out
of 64 accepted papers (48\%), POPL 2016 had 25 artifacts submitted out of 59
accepted papers (42\%), and POPL 2015 had 29 artifacts submitted out of 52
accepted papers (56\%).

This is the second year when the POPL AEC has decided to not accept paper proofs
in the artifact evaluation process. The AEC lacks the time and often the
expertise to carefully review paper proofs. We also hope that reserving the
artifact evaluated badge to mechanized proofs that are easy to check and reuse
will incentivize more of the POPL authors to mechanize their metatheory in a
proof assistant. Paper proofs can still be uploaded to the ACM Digital Library
together with the final version, and this policy seems to have caused no strong
objections so far.

The schedule for artifact evaluation was very tight this year: the AEC only had
a little under two weeks to evaluate the artifacts. The AEC members were given
the deliberately vague instruction to evaluate whether the artifact met the
expectations set by the paper. The AEC also purchased Amazon EC2 server
instances in order to give reviewers access to more computational resource.
(This year we spent \$150 on this.)

One novelty this year is that with the help of Eddie Kohler we set up the HotCRP
conference management system so that the anonymous reviewers could ask questions
using author-visible comments and get immediate answers from the authors. So
reviewers who get stuck on installing or trying an artifact or who noticed
something that looked fishy with an artifact could directly and anonymously ask
the authors for help or clarifications directly from the HotCRP interface. This
new feature that Eddie Kohler added to HotCRP specially for us worked as
expected and was greatly appreciated by both reviewers and authors alike, so we
recommend continuing to use it in the next years. At the same time, turning on
this feature in HotCRP meant not only that the author-visible comments were
immediately visible to the authors but also the reviews. While the reviewers
were made aware of this from the start, it did seem to confuse some of the
authors to get the reviews as they were ready. Some of the reviewers also found
it difficult to change their scores after the reviews were sent out, so ideally
we would like that only explicitly marked comments be sent to authors
immediately, but not the reviews, for which we could have a more traditional
rebuttal period if time permits. Supporting this would require more work in
HotCRP though and a less tight schedule to allow for rebuttals.

The average AEC reviewer submitted 2.9 reviews and all reviews were submitted on
time, which is a big achievement given the extremely tight schedule. After this,
the AEC held a brief online discussion to decide, for each artifact, if it met,
exceeded, or fell below the expectations set by its paper. The scores were to a
very large extent positive (the average merit score was 4.15 out of 5) so there
were few debates. In the end all 40 submitted artifacts were judged to meet or
exceed expectations. The 100\% acceptance rate is quite remarkable. We believe
this has to do on the one hand with the fact that the artifact evaluation
process is now more streamlined, so authors have a better idea how to prepare
good or at least acceptable artifacts. On the other hand, this seems to also be
caused by allowing the anonymous reviewers to directly interact with the authors
to sort out any issues. Fast feedback and dialogue helped reviewers avoid
frustration and unduly harsh reviews.

After the discussion phase, artifact decisions were communicated to authors
almost one week before the final versions of papers were originally due, but the
final version deadline was subsequently extended. With this extension at least,
this timeline allowed authors sufficient time to incorporate feedback from the
AEC reviewers in the final versions of papers and artifacts.

The papers that describe accepted artifacts can be recognized by ACM’s new
``Artifacts Evaluated - Functional'' badge. After discussions with the AEC chairs
of the other PACMPL conferences we decided that POPL’s old badges were closest
to ACM’s new “Artifacts Evaluated - Functional” badges so we simply replaced our
old badges with this for now. The ACM also provides more valuable “Artifacts
Evaluated - Reusable” badges though, that we could decide to also use in the
future for ``distinguished artifacts'', at the cost of moving to a 2-tier system
that would be more work for the AEC, who would have to make this distinction.

We additionally assigned the ``Artifacts Available'' badge to papers that passed
artifact evaluation and where the authors also make their artifacts publicly
available eternally on the ACM Digital Library. We were assured that the only
permissions the authors need to grant the ACM is non-exclusive rights to
distribute the artifact and that such artifacts will be available in perpetuity
for free to anyone, without any paywall. The requirement to use the ACM Digital
Library for getting this badge is still a bit debatable given that the ACM
policy explicitly mentions that the ACM does ``not mandate the use of specific
repositories''. Yet the immutability and permanence constraints the ACM imposes
are very strong and they currently get interpreted as having to use ACM DL for
getting the “Artifacts Available” badge. Saving a permanent snapshot of their
artifacts to the ACM DL does not in any way prevent authors to also make them
available in other ways, and we encourage authors to additionally make their
artifacts open source on platforms like GitHub, and are considering having that
as a requirement in the future for the cooler ``Artifacts Evaluated - Reusable''
badges.

We thank the authors of the all submitted artifacts for their work in preparing
and documenting their research output. We hope that they found the feedback from
the AEC to be helpful. And last but not least, we are immensely grateful to the
members of the AEC for their energy and enthusiasm.

\begin{flushright}
\textit{C\u{a}t\u{a}lin Hri\c{t}cu, Inria Paris} \\
POPL 2018 Artifact Evaluation Committee Co-Chair
\medskip \\
\textit{Jean Yang, Carnegie Mellon University} \\
POPL 2018 Artifact Evaluation Committee Co-Chair
\medskip \\
\end{flushright}

\newpage

%% POPL 2018 AEC members
%% Sara Achour, MIT
%% Ales Bizjak, Aahus University
%% Joachim Breitner, University of Pennsylvania
%% Stefan Ciobaca, Alexandru Ioan Cuza University of Iasi
%% Michael Coblenz, Carnegie Mellon University
%% Santiago Cuellar, Princeton
%% Stefan Fehrenbach, University of Edinburgh
%% Yu Feng, University of Texas at Austin
%% Paul Gazillo, Yale University
%% Nick Giannarakis, Princeton University
%% Klaus Gleissenthall, University of California, San Diego
%% Fredrik Kjolstad, MIT
%% William Mansky, Princeton University
%% Ruben Martins, Carnegie Mellon University
%% Jedidiah McClurg, University of Colorado Boulder
%% Sergio Mover, University of Colorado Boulder
%% Daejun Park, University of Illinois at Urbana-Champaign
%% Jennifer Paykin, University of Pennsylvania
%% Clement Pit-Claudel, MIT
%% Giselle Reis, Carnegie Mellon University
%% Talia Ringer, University of Washington
%% Taro Sekiyama, IBM Research, Japan
%% Jan Stolarek, University of Edinburgh
%% Pierre-Yves Strub, Ecole Polytechnique
%% Martin Suda, TU Wien
%% Andrea Vandin, DTU
