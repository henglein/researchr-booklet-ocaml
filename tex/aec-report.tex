\header{AEC Chairs' Report}{}{}{Artifact Evaluation Chairs' Report}
\label{aec}

In the programming languages community, artifact evaluation is concerned with
the byproducts of theoretical and applied work. An ``artifact'' is something
intended to support the scientific claims made in a paper. For instance, an
artifact might be a program’s source code, a dataset, a test suite, a mechanized
proof, or a model. ``Evaluation'' is a best-effort attempt to reconcile a paper’s
artifacts with the claims made in the paper. A primary goal of the artifact
evaluation process is to encourage authors to create artifacts that can be
shared and used by others as a basis for new activities. The process has other
benefits as well, such as encouraging authors to be precise in their claims and
providing public recognition for the effort invested in creating artifacts. To
encourage this beneficial behavior, since POPL 2015, authors of accepted papers
have been invited to submit artifacts for evaluation by an Artifact Evaluation
Committee (AEC). In this report we describe the process of artifact evaluation
for POPL 2019, and discuss what worked well and what can still be improved.

Members of the POPL Program Committee were invited to nominate senior PhD
students, PostDocs, researchers, or themselves to serve on the AEC.
Participation in the AEC can provide researchers with useful insight into the
value of artifacts and the process of artifact evaluation, and also helps
establish community norms for artifacts. Starting from the nominations and after
some targeted search aimed at balancing the AEC, the chairs formed a committee
of 30 members.

After the POPL 2019 decisions were announced, authors of conditionally accepted
papers were invited to submit artifacts for evaluation. By design, the artifact
evaluation process had no effect on which papers were chosen to appear at POPL.
Authors had one week to respond to the call for artifacts and submit their
artifact for evaluation.
% The submission guidelines asked that the authors create a single web
% page that contains the artifact and instructions for installing and
% using the artifact, while allowing for some exceptional cases.
Each artifact was accompanied by the associated paper so that the AEC
could evaluate each artifact against its paper's claims. A record
number of 43 artifacts were submitted for evaluation out of 77 papers
accepted this year (56\%). For comparison,
POPL 2018 had 40 artifacts submitted out of 66 conditionally accepted (61\%),
POPL 2017 had 31 artifacts submitted out of 64 accepted papers (48\%),
POPL 2016 had 25 artifacts submitted out of 59 accepted papers (42\%), and
POPL 2015 had 29 artifacts submitted out of 52 accepted papers (56\%).

% This is the third year when the POPL AEC has decided to not accept paper proofs
% in the artifact evaluation process. The AEC lacks the time and often the
% expertise to carefully review paper proofs. We also hope that reserving the
% artifact evaluated badge to mechanized proofs that are easy to check and reuse
% will incentivize more of the POPL authors to mechanize their metatheory in a
% proof assistant. Paper proofs can still be uploaded to the ACM Digital Library
% together with the final version, and this policy seems to have caused no strong
% objections so far.

% The schedule for artifact evaluation was again tight this year: the
% AEC only had only two weeks to evaluate whether the artifacts met the
% expectations set by the paper.
%
The AEC members did an amazing job at evaluating artifacts and
providing helpful feedback.
%
The average AEC reviewer submitted 2.7 reviews and the large majority
of reviews were submitted on time, which is a big achievement given
the tight schedule.
%
After this, the AEC held an online discussion to decide, for each
artifact, if it met, exceeded, or fell below the expectations set by
its paper.
%
The scores were to a very large extent positive (the average merit
score was 3.99 out of 5).
%
In the end 42 out of 43 submitted artifacts were judged to meet or
exceed expectations (97.7\%).
%
We believe that artifact evaluation is simply confirming that the POPL
community is great at producing high quality artifacts.

Beyond the high quality of artifacts we receive, another reason we could
achieve such a high acceptance rate is that since last year we set up
the HotCRP conference management system so that the anonymous
reviewers could ask questions using author-visible comments and get
immediate answers from the authors. In addition, any finalized reviews
are immediately made available to the authors, and authors can comment
on those reviews as well. This was a new feature that Eddie Kohler
added to HotCRP last year specially for us, so reviewers who get stuck
on installing or trying an artifact or who noticed something that
seems fishy with an artifact could directly and anonymously ask the
authors for help or clarifications or fixes using the HotCRP
interface. Feedback from reviewers about this feature was
overwhelmingly positive, so we plan to keep doing this in the
future and we warmly recommend it to the AECs of other conferences.

As always there are things which could be improved. One struggle we
seem to have each year is getting enough time for artifact
evaluation. Both this and last year the committee only had two weeks
for the actual reviewing. This works in the limit, but it adds quite a
bit of stress. And even getting these two weeks required hard work,
since the publishing process was revamped for this edition of POPL,
and we had to ensure that the AEC feedback makes it to the authors in
time to still make changes to the paper (e.g. if the paper sets
certain inaccurate expectations about the artifacts, which is a not so
uncommon complaint of AEC reviewers).

Then the recent changes of the badging system imposed by the ACM
brought both issues and opportunities. AT POPL we deployed these
changes incrementally: last year we rolled out the “Artifacts
Available” badge and this year we rolled out the “Functional” vs
“Reusable” distinction.

The “Artifacts Available” badge we implemented last year seemed easy,
until we realized somewhere late in the process that the ACM policy
that the publisher wanted to follow would blindly give this badge to
every paper for which the authors uploaded any materials to the ACM
DL, completely ignoring the artifact evaluation process. Since we
didn’t want to water down the AEC badging process with such a lax
policy, last year we escalated this to the PACMPL level, where the AEC
chairs of the other conferences agreed we should give the “Artifacts
Available” badge only to papers that successfully passed artifact
evaluation and where the authors additionally upload their evaluated
artifact (not random stuff) to the ACM DL. This became the policy by
which we gave out the “Artifact Available” badges at POPL for the last
two years. We had no complaints about this restriction to the ACM
policy, and this year 88\% of the papers with successfully accepted
artifacts got this extra badge. The only remaining (minor) issue is
making the publishers improve their online forms so that uploads of
AEC-accepted artifacts are clearly separated from uploads of other
random materials, so that we can reduce unnecessary guesswork in
awarding this badge.

Implementing a distinction between the “Functional” and “Reusable”
this year was trickier, since the official ACM guidelines were rather
vague. We came up with a simple policy that incentivizes authors to:
(1) make their artifacts as good as possible and (2) release their
artifacts under an open source license. We felt that point (1) is
useful, since with close to 100\% acceptance rate the authors of a
good enough artifact might not have much incentive to package and
document it well and in general do the extra work required to make
their artifact great. Point (2) is useful since having the source code
available under an open source license makes easy reuse by others
possible (at least in principle), while the “Artifacts Available”
badge is happily given to artifacts that are only released as a binary
blob and/or under onerous licenses, as long as they are permanently
archived. A pleasant side effect of having two levels of badges was
that it encouraged more discussions in order to sort the
artifacts. Last year, our very high average scores limited discussions
to the few artifacts with the lowest scores.


After a couple of quick iterations, we ended up with the following
text for our policy: “The AEC will offer the ‘Artifacts Evaluated -
Reusable’ badge to the artifacts that receive above average scores and
that are additionally made available in a way that enables reuse (for
software artifacts this means releasing the artifact under an open
source license, ideally on a platform like GitHub, GitLab, BitBucket,
SourceForge, etc). All other artifacts that meet expectations will get
the default ‘Artifacts Evaluated - Functional’ badge.” From the past
year’s data, we were able to predict that the average would be around
4.0 (which corresponds to “Accept” on the usual 5-point scale in
HotCRP), and we communicated to the reviewers from the beginning that
artifacts with a score of 4.0 or higher will be good candidates for
the Reusable badge. Our average score this year turned out 3.99. Most
artifacts with an average score of 4.0 or higher that were made open
source were given the Reusable badge, with leeway for the reviewers to
decide when the score was 4.0. In the end 69\% of the accepted
artifacts got the Reusable badge and 31\% got the Functional badge.


Our impression is that this simple policy worked very well this year:
93\% of the submitted artifacts were released as open source, which
helps reusability. Moreover, many authors put serious energy into
quickly answering questions and quickly fixing, improving, or
documenting their artifacts in response of reviewer feedback, even
when it was clear that their artifact would anyway be accepted, so the
“Reusable” badge seems to incentivize authors to produce better
artifacts. Finally, all the artifacts saw at least some discussion
this year to decide between the two badges, which was great progress
from last year’s limited discussions. Beyond the increase in
discussions, we don’t think that making this new distinction using the
simple policy above made the review process more difficult for the
reviewers.


So, we managed to make good use of the changes of the badging system
imposed by the ACM to further incentivize artifact authors to produce
better and more reusable artifacts. We’re sure one can have
philosophical discussions whether there are better ways one can better
interpret the “Available”, “Functional”, and “Reusable” designations
that the ACM threw at us, along with their strange policies for
awarding these badges. We feel that the simple policies we have
devised for POPL have the advantage of being easy to understand, easy
to enforce, and easy to integrate into our already well-working
artifact evaluation process. That’s why we appreciated trying this out
incrementally on our own, and we believe this positive experience can
inform potential future standardization efforts at the SIGPLAN level.

We thank the authors of the all submitted artifacts for their work in
preparing and documenting their research output. We hope that they
found the feedback from the AEC to be helpful. And last but not least,
we are immensely grateful to the members of the AEC for their energy
and enthusiasm.

\begin{flushright}
\textit{Benjamin Delaware, Purdue University} \\
\textit{C\u{a}t\u{a}lin Hri\c{t}cu, Inria Paris} \\
POPL 2019 Artifact Evaluation Committee Chairs
\medskip \\
\end{flushright}

\newpage
