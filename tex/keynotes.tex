\header{Keynotes}{}{}{POPL Keynotes}
\label{Keynotes}

\def\talktitle#1{\subsection*{#1}}
\def\speaker#1#2{\begin{flushleft} #1 (#2) \end{flushleft}}
\def\talkabstract{\noindent \textbf{Abstract:}~}
\def\bio{\medskip\noindent \textbf{Bio:}~}

\talktitle{Automated Fault-Finding and Fixing at Facebook}

\speaker{Mark Harman}{Facebook and University College London}

\talkabstract
This talk will describe the deployment of Sapienz, a system for automated test
case design that uses Search Based Software Engineering (SBSE) and which has
been deployed at Facebook since October 2017 to design test cases, localise
and triage faults to developers and to monitor their fixes (and, since August
2018, also to automate the process of fixing some of these faults). The talk
will also cover the way in which we combine static and dynamic analysis for
fault-finding and fixing at Facebook, concluding with a discussion of some
open problems for static and dynamic analysis and challenges in deploying them
at scale. This keynote is an account of the joint work of the whole Sapienz
team and its partners and collaborators at Facebook.

\bio
Mark Harman is an engineering manager at Facebook London, where he manages the
Sapienz team, working on Search Based Software Engineering (SBSE) for
automated test case design and fault fixing. Sapienz has been deployed to
continuously test Facebook’s Android and iOS apps, leading to thousands of
bugs being automatically found and fixed (mostly by developers, but more
recently, some of these faults have also been automatically fixed by
Sapienz). The software tackled by Sapienz consists of tens of millions of
lines of code; apps that are among the largest and most complex in the app
store and that are used by over a billion people worldwide every day for
communication, social networking and community building. Mark is also a part
time professor of Software Engineering in the Department of Computer Science
at University College London, where he directed the CREST centre for ten years
(2006-2017) and was Head of Software Systems Engineering (2012-2017). He is
known for his scientific work on SBSE, source code analysis, software testing,
app store analysis and empirical software engineering. He was the co-founder
of the field SBSE, which has grown rapidly with over 1,700 scientific
publications from authors spread over more than 40 countries. SBSE research
and practice is now the primary focus of his current work in both the
industrial and scientific communities. In addition to Facebook itself, Mark’s
scientific work is also supported by an ERC advanced fellowship grant and by
the UK EPSRC funding council.

\newpage

\talktitle{Mechanized Metatheory - The Next Chapter}

\speaker{Brigitte Pientka}{McGill University}

\talkabstract Mechanizing formal systems and proofs about them plays an
important role in establishing trust in programming languages and verifying
software systems in general. Over the past decades, we have seen significant
progress and success: the POPLMark challenge popularized the use of proof
assistants in mechanizing the metatheory of programming languages; the
development of the first verified compiler in Coq demonstrated the feasibility
and value of building verified compilers. Today, mechanizing proofs is a
stable fixture in the daily life of programming languages researchers.

One might be tempted to assume that all is well – yet, the reality is that we
seem to accept the status quo and focus on engineering solutions to get the
next job done, although mechanizations can be time consuming and require
substantial (sometimes heroic) effort. In this talk I will argue that we
should strive to develop high-level abstractions, primitives, and tools to
standardize common tasks and operations arising when working with formal
systems and proofs about them.

Taking a look back at some of the foundational theoretical ideas and choices
that underly type-theoretic proof environments today, I will survey recent
achievements and sketch a dependently typed language that would make it easier
to represent, maintain, and communicate formal systems and proofs about them –
an elegant proof language for a more civilized age.


\bio Brigitte Pientka is an Associate Professor in the School of Computer
Science at McGill University, and leading the Computation and Logic group. She
received her PhD from Carnegie Mellon University in 2003, and studied
previously at the University of Edinburgh and Technical University of
Darmstadt.

\newpage
