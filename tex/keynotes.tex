\header{Keynotes}{}{}{POPL Keynotes}
\label{Keynotes}

\def\talktitle#1{\subsection*{#1}}
\def\speaker#1#2{\begin{flushleft} #1 (#2) \end{flushleft}}
\def\talkabstract{\noindent \textbf{Abstract:}~}
\def\bio{\medskip\noindent \textbf{Bio:}~}

\talktitle{The Type Soundness Theorem That You Really Want to Prove (and Now You Can)}

\speaker{Derek Dreyer}{MPI-SWS}

\talkabstract
Type systems—and the associated concept of “type soundness”—are one of the
biggest success stories of foundational PL research. Originally proposed by
Robin Milner in 1978, type soundness asserts that well-typed programs can’t “go
wrong” (\emph{i.e.}, exhibit undefined behaviors), and it is widely viewed as the
canonical theorem one must prove to establish that a type system is doing its
job. In the early 1990s, Wright and Felleisen introduced a simple syntactic
approach to proving type soundness, which was subsequently popularized as the
method of ``progress and preservation'' and has had a huge impact on the study and
teaching of PL foundations. Many research papers that propose new type systems
conclude with a triumphant statement of syntactic type soundness, and for many
students it is the only thing they learn to prove about a type system.

Unfortunately, syntactic type soundness is a rather weak theorem. First of all,
its premise is too strong for many practical purposes. It only applies to
programs that are completely well-typed, and thus tells us nothing about the
many programs written in ``safe'' languages that make use of ``unsafe'' language
features. Even worse, it tells us nothing about whether type systems achieve one
of their main goals: enforcement of data abstraction. One can easily define a
language that enjoys syntactic soundness and yet fails to support even the most
basic modular reasoning principles for closures, objects, and ADTs.

In this talk, I argue that we should no longer be satisfied with just proving
syntactic type soundness, and should instead start proving a stronger
theorem—semantic type soundness—that captures more accurately what type systems
are actually good for. In a semantic soundness proof, one defines a semantic
model of types as predicates on values, and then verifies the soundness of
typing rules as lemmas about the model. By explaining directly what types
``mean'', the semantic approach to type soundness is a lot more informative than
the syntactic one. In particular, it can serve to establish what data
abstraction guarantees a language provides, as well as what it means for uses of
unsafe language features to be ``safely encapsulated''.

Semantic type soundness is a very old idea—Milner’s original formulation of type
soundness was a semantic one—but it fell out of favor in the 1990s due to
limitations and complexities of denotational models. In the succeeding decades,
such limitations have been overcome and complexities tamed, via proof techniques
that work directly over operational semantics. Thanks to the development of
step-indexed Kripke logical relations, we can now scale semantic soundness to
handle real languages, and thanks to advances in higher-order concurrent
separation logic, we can now build (machine-checked) semantic soundness proofs
at a much higher level of abstraction than was previously possible. The
resulting ``logical'' approach to semantic type soundness yields proofs that are
demonstrably more useful than their syntactic counterparts, and more fun as
well.

\bio
Derek Dreyer is a professor of computer science at the Max Planck Institute for
Software Systems (MPI-SWS), and recipient of the 2017 ACM SIGPLAN Robin Milner
Young Researcher Award.  His research runs the gamut from the type theory of
high-level functional languages, down to the verification of compilers and
low-level concurrent programs under relaxed memory models.  He is currently
leading the RustBelt project, which focuses on building the first formal
foundations for the Rust programming language.  He also knows a thing or two
about Scotch whisky.

\medskip

\talktitle{Some Principles of Differential Programming Languages}

\speaker{Gordon Plotkin}{University of Edinburgh}

\talkabstract
TODO

\bio
Gordon David Plotkin, FRS, FRSE (born 9 September 1946) is a theoretical
computer scientist in the School of Informatics at the University of Edinburgh.
Plotkin is probably best known for his introduction of structural operational
semantics (SOS) and his work on denotational semantics. In particular, his notes
on A Structural Approach to Operational Semantics were very influential.

Plotkin was elected a Fellow of the Royal Society in 1992, is a Fellow of the
Royal Society of Edinburgh and a Member of the Academia Europaea. He is also a
winner of the Royal Society Wolfson Research Merit Award. Plotkin received the
2012 Royal Society Milner Award for ``his fundamental research into programming
semantics with lasting impact on both the principles and design of programming
languages.''

\medskip

\talktitle{Formal Methods and the Law}

\speaker{Sarah Lawsky}{Northwestern Pritzker School of Law}

\talkabstract
TODO

\bio
Sarah Lawsky is Professor of Law at Northwestern Pritzker School of Law. She
teaches or has taught federal income tax, corporate tax, partnership tax, tax
policy, tax deals, and contracts. Her research focuses on tax law and on the
application of formal logic and artificial intelligence to the law.

Prior to joining Northwestern Pritzker in 2016, Lawsky taught at UC Irvine
School of Law and George Washington University Law School, and as an adjunct in
NYU’s tax LL.M. program. Before beginning her teaching career, she practiced tax
law in New York.

Lawsky received her B.A. from the University of Chicago, her J.D. from Yale Law
School, her LL.M. in tax from NYU School of Law, and her Ph.D. in philosophy
from the UC Irvine Department of Logic and Philosophy of Science.

\newpage
