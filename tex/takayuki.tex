\header{In memoriam}{}{}{Takayuki Muranushi (1983--2017)}
\label{Takayuki}

\noindent
It is with great sadness that we have to report of the untimely
passing of the CUFP Tutorial Co-Chair Takayuki Muranushi. We have lost
an extraordinarily promising researcher and a true champion of
functional programming.

Takayuki Muranushi was not trained as functional programmer: born in
1983, he entered Kyoto University in the physics and cosmology
program. His PhD thesis (defended in 2013 at Kyoto University's
Institute for Theoretical Physics) was titled ``Lightning in
Protoplanetary Disks''. Writing code -- in Fortran or, nowadays, C++,
is part of contemporary physics practice. That was not enough for Takayuki. He
went one level up: he got the computer to write C++ physics code,
that was faster than any code a human could write, or comprehend.

I caught a glimpse of this metaprogram, Paraiso, from Takayuki's
presentation at the 2012 Shonan meeting on code generation for
high-performance computing, where I first met him. It is a Haskell
program that takes a partial differential equation in familiar
mathematical notation and, relying on sophisticated genetic
algorithms, generates C/C++ code for a GPGPU or a multicore CPU.  His
aspiration, close to fulfillment, was to make Paraiso the standard for
partial differential equations, as FFTW is for fast Fourier transforms.
I was very impressed by the sophistication of Paraiso -- and also by a
rarely seen enthusiastic presentation of it. He had managed to fit 125
slides within allotted 10 minutes -- in a way that I still remember
them five years later.

Takayuki continued this work after graduation, at RIKEN Advanced
Institute for Computational Science in the Particle Simulator Research
Team. Last year he with his teammates presented a paper at the
ICFP-affiliated FHPC workshop, on automatic generation of efficient code from
mathematical descriptions of stencil computation. He has also
co-authored several papers at the Haskell Symposium, and this year served
on its Program Committee.

Takayuki not only studied functional programming: he lived it.  In
2006, right after finishing his undergraduate studies, he co-founded a
start-up `Preferred Infrastructure (PFI)'. When I visited that company
once, everyone I met was eager to tell me that although `Preferred
Infrastructure' is the official company name, what PFI really stands
for is `Pure Functional Inc.' -- the name given by Takayuki. He is
most well-known in Japan for the Japanese edition of Miran Lipovaca's book `Learn You a
Haskell', which he co-translated with his PFI colleague Hideyuki Tanaka.

He was so brilliant that he seemed not of this world. And now he
isn't.  His papers, his code, his automatic (robotic) book scanner,
his book continue to surprise and delight.  Those of us who were
fortunate to have met him shall remember, and sorely miss him.

\begin{flushright}
  -- \textit{Oleg Kiselyov}
\end{flushright}

\newpage

